\batchmode
\documentclass[
%8pt, 9pt, 10pt, 11pt, 12pt, 14pt, 17pt, 20pt
%serif,
%table, % for table coloring
%draft,
%ngerman,
%handout,	% remove overlays
compress,
xcolor=table,
dvipsnames,
]{beamer}

\usepackage{tipa}
%% Encoding, fonts, language
\input{settings/pdflatex_setup}

\usepackage{etex} 
\usepackage{graphics}
\usepackage{subcaption}
\input{settings/tikz_setup} 
\input{settings/forest_setup}

\usepackage{url}
\usepackage{amsmath,amssymb,amsfonts,marvosym}
\usepackage{ulem}			% to cross out text
\normalem

\usepackage{ragged2e}
\let\raggedright=\RaggedRight

%\usepackage[utf8]{inputenc} %Este es el paquete para que te muestre bien los caracteres latinos
%\usepackage[useregional]{datetime2} %paquete para la fecha

\usepackage{linguex}   % must be loaded below \usepackage[T1]{fontenc}
\AtBeginDocument{
  \setlength{\Exlabelsep}{0em}		% for linguex examples
  \setlength{\SubExleftmargin}{1,5em}	% for linguex examples
  \renewcommand\eachwordone{\sffamily}	% for glossing with linguex
  \renewcommand\eachwordtwo{\sffamily}	% for glossing with linguex
  % \setlength{\Extopsep}{1ex}   % vertical margin in linguex examples
}
\input{settings/avm_setup}

\input{settings/beamer_setup}

%% Bibliography
%% BibLaTeX
\input{settings/biblatex_setup.tex}
%% BibTeX 
% \input{settings/natbib_bibtex_setup.tex} 
% \PassOptionsToPackage{round}{natbib}
% \renewcommand{\newblock}{}    % to make natbib compatible with beamer


%%%%%%%%%%%%%%%%%%%%%%%%%%%%%%% 
%    non-essential MACROS     %
%%%%%%%%%%%%%%%%%%%%%%%%%%%%%%%

\input{myMacros}
\definecolor{myblue}{rgb}{0,0,0.70}
\definecolor{myred}{rgb}{0.8,0,0}
\definecolor{mydarkgreen}{rgb}{0,0.55,0}



%%%%%%%%%%%%%%%%%%%%%%%%%%%%%%%%%%%%%%%%%%%%%%%%%%%%%%%%%%%%%%%%%%%%%%%%%%%%%
% HEADER
%%%%%%%%%%%%%%%%%%%%%%%%%%%%%%%%%%%%%%%%%%%%%%%%%%%%%%%%%%%%%%%%%%%%%%%%%%%%%

\title[\arabic{page} ]{Fundamentos Teóricos de Informática}
\subtitle[RP]{Redes de Petri}	
\author{Luciano Serruya Aloisi}
\institute[UNPSJB]{Universidad Nacional de la Patagonia San Juan Bosco}
\date{
    Cátedra: \endgraf 
    Dra. Celia Cintas\endgraf 
    Lic. Pablo Navarro\endgraf 
    Lic. Samuel Almonacid\endgraf
    \medskip
    \medskip 
    22 de Febrero del 2018
}

\logo{\pgfimage[width=0.8cm,height=1cm]{graphics/logoUnpsjb}}			% Logo on all slides (pdf,png,jpg,eps)
%\titlegraphic{\includegraphics[height=3cm]{graphics/logoUnpsjb}}	% Logo on title slide


%%%%%%%%%%%%%%%%%%%%%%%%%%%%%%%%%%%%%%%%%%%%%%%%%%%%%%%%%%%%%%%%%%%%%%%%%%%%%
% SLIDES
%%%%%%%%%%%%%%%%%%%%%%%%%%%%%%%%%%%%%%%%%%%%%%%%%%%%%%%%%%%%%%%%%%%%%%%%%%%%%

\begin{document}

\begin{frame}[plain]
    \begin{titlepage}
        
    \end{titlepage}
\end{frame}

%\frame{\titlepage}

%\frame{
%\frametitle{Temario}
  %\tableofcontents
  %%[pausesections]
%}

%\AtBeginSection[]
%{
%  \begin{frame}<beamer>{Outline}
%    \tableofcontents[
%    currentsection
%    ]
%  \end{frame}
%}

%%%%%%%%%%%%%%%%%%%%%%%%%%%%%%%%%%%%%%%%%%%%%%%%%%%%%%
\begin{frame}{Introducción}
    \begin{itemize}
        \item Qué son y para qué sirven
        \item{Definición}
        \item{Marcados y relación de orden}
        \item{Transiciones habilitadas y disparos}
        \item{Secuencias de disparos}
    \end{itemize}
\end{frame}
%%%%%%%%%%%%%%%%%%%%%%%%%%%%%%%%%%%%%%%%%%%%%%%%%%%%%%

\begin{frame}{Ejemplo: filósofos comensales}
    \begin{figure}[h]
        \includegraphics[scale=0.45]{graphics/dining_philosophers.png}
        \caption{Filósofos comensales \citep{Labri}}
    \end{figure}
\end{frame}

%%%%%%%%%%%%%%%%%%%%%%%%%%%%%%%%%%%%%%%%%%%%%%%%%%%%%%

\begin{frame}{Modelos}
    \begin{itemize}
        \item{Autómatas finitos}
        \item{Actividades paralelas}
        \item{Computación de flujo de datos}
        \item{Protocolos de comunicación}
        \item{Control de sincronización}
        \item{Productor-Consumidor}
        \textbf{\item{Lenguajes formales}}
        \item{Sistemas multiprocesador}
    \end{itemize}
    \begin{itemize}
        \item Concurrencia
        \item Conflictos
        \item Confusión
            \begin{itemize}
                \item Simétrica
                \item Asimétrica
            \end{itemize}
    \end{itemize}
\end{frame}

%%%%%%%%%%%%%%%%%%%%%%%%%%%%%%%%%%%%%%%%%%%%%%%%%%%%%%

\begin{frame}{Autómatas finitos}
    \begin{figure}[h]
        \includegraphics[scale=0.8]{graphics/af_rp.png}
        \caption{Autómata finito \citep{Murata:89}}
    \end{figure}
\end{frame}

%%%%%%%%%%%%%%%%%%%%%%%%%%%%%%%%%%%%%%%%%%%%%%%%%%%%%%

\begin{frame}{Conflictos y concurrencia}
  \begin{columns}
    \column{.42\textwidth}
    \begin{figure}[h]
        \includegraphics[scale=0.7]{graphics/rp_conflicts.png}
        \caption{Conflictos en RP \citep{Murata:89}}
    \end{figure}
    \column{.58\textwidth}
        \begin{figure}[h]
            \includegraphics[scale=0.7]{graphics/rp_concurrency.png}
            \caption{Concurrencia en RP \citep{Murata:89}}
        \end{figure}
  \end{columns}
\end{frame}

%%%%%%%%%%%%%%%%%%%%%%%%%%%%%%%%%%%%%%%%%%%%%%%%%%%%%%

\begin{frame}{Confusión}
  \begin{columns}
    \column{.50\textwidth}
    \begin{figure}[h]
        \includegraphics[scale=1]{graphics/symmetric_confusion.png}
        \caption{Confusión simétrica \citep{Murata:89}}
    \end{figure}
    \column{.50\textwidth}
        \begin{figure}[h]
            \includegraphics[scale=1]{graphics/asymmetric_confusion.png}
            \caption{Confusión asimétrica \citep{Murata:89}}
        \end{figure}
  \end{columns}
\end{frame}

%%%%%%%%%%%%%%%%%%%%%%%%%%%%%%%%%%%%%%%%%%%%%%%%%%%%%%

\begin{frame}{Métodos de análisis}
    \begin{itemize}
        \item Árbol de cobertura
        \item Matriz de incidencia
    \end{itemize}
\end{frame}

%%%%%%%%%%%%%%%%%%%%%%%%%%%%%%%%%%%%%%%%%%%%%%%%%%%%%%

\begin{frame}{Árbol de cobertura - Marcados frontera}
    \begin{itemize}
        \item Marcados muertos
        \item Nodos duplicados
        \item Marcados que sólo se diferencian del anterior por tener un número distinto de \textit{tókens} en alguna plaza y que habilita el mismo conjunto de transiciones. Se remplaza la plaza con la cantidad de tókens distinta por el símbolo $\omega$
    \end{itemize}
\end{frame}

%%%%%%%%%%%%%%%%%%%%%%%%%%%%%%%%%%%%%%%%%%%%%%%%%%%%%%

\begin{frame}{Árbol de cobertura}
  \begin{columns}
    \column{.50\textwidth}
    \begin{figure}[h]
        \includegraphics[scale=0.3]{graphics/rp_inifite.png}
        \caption{RP con infinitos marcados \citep{Murata:89}}
    \end{figure}
    \column{.50\textwidth}
        \begin{figure}[h]
            \includegraphics[scale=0.35]{graphics/reachability_tree.png}
            \caption{Árbol de cobertura \citep{Murata:89}}
        \end{figure}
  \end{columns}
\end{frame}

%%%%%%%%%%%%%%%%%%%%%%%%%%%%%%%%%%%%%%%%%%%%%%%%%%%%%%

\begin{frame}{Matriz de incidencia}

    Sea una RP = (P, T, I, O) con \textit{n} plazas y \textit{m} transiciones, se define la \textbf{matriz de incidencia previa} como: 
    \begin{equation}
        C^-(j, i) = I(p_i, t_j), \quad 1 \leq j \leq m, \quad 1 \leq i \leq n  
    \end{equation}
    Se define la \textbf{matriz de incidencia posterior} como:
    \begin{equation}
        C^+(j, i) = O(t_j, p_i), \quad 1 \leq j \leq m, \quad 1 \leq i \leq n  
    \end{equation}
    Se define la \textbf{matriz de incidencia} como:
    \begin{equation}
        C = C^+ - C^-
    \end{equation}
\end{frame}

%%%%%%%%%%%%%%%%%%%%%%%%%%%%%%%%%%%%%%%%%%%%%%%%%%%%%%

\begin{frame}{Representación matricial}

    Una transición $t_j$ se define por un vector $e_j$ de dimensión \textit{m} de componentes:

        \[ e_j(i) =
          \begin{cases}
            1       & \text{si } i = j\\
            0       & \text{si } i \neq j
          \end{cases}
        \]
    Una transición $t_j$ está habilitada en un marcado \textit{M} si
    \begin{equation}
        M \geq e_j \bullet C^-
    \end{equation}
    El resultado del disparo de la transición $t_j$ a partir del estado \textit{M} es:
    \begin{equation}
        M' = M + e_j \bullet C
    \end{equation}

\end{frame}

%%%%%%%%%%%%%%%%%%%%%%%%%%%%%%%%%%%%%%%%%%%%%%%%%%%%%%

\begin{frame}{Propiedades de comportamiento}
    \begin{itemize}
        \item{Alcanzabilidad}
        \item{Límite}
            \begin{itemize}
                \item{Seguridad}
            \end{itemize}
        \item{Vitalidad}
        \item{Estado base}
        \item{Cobertura}
        \item{Persistencia}
        \item{Justicia}
    \end{itemize}
\end{frame}

%%%%%%%%%%%%%%%%%%%%%%%%%%%%%%%%%%%%%%%%%%%%%%%%%%%%%%

\begin{frame}{Lenguajes formales y RP}
    \begin{itemize}
        \item{Clases}
        \item{Expresividad}
    \end{itemize}
\end{frame}

%%%%%%%%%%%%%%%%%%%%%%%%%%%%%%%%%%%%%%%%%%%%%%%%%%%%%%

\begin{frame}{Definición de los lenguajes RP}
    Para que una RP pueda ser reconocedora de lenguajes, se deben definir tres conceptos más:
    \begin{itemize}
        \item{Estado inicial}
        \item{Etiquetado $ \sigma: T \mapsto \Sigma$}
        \item Estado final
    \end{itemize}
\end{frame}

%%%%%%%%%%%%%%%%%%%%%%%%%%%%%%%%%%%%%%%%%%%%%%%%%%%%%%

\begin{frame}{Conjunto de estados finales}
    Según cómo se defina el conjunto de estados finales de una RP reconocedora, el lenguaje reconocido será distinto \citep{Peterson:81}
    \begin{itemize}
        \item{Tipo-L = $\{\sigma(\beta) \in \Sigma^*, \quad \beta \in T^* \text{ y }\delta(M_0, \beta) \in F\}$}
        \item{Tipo-G = $\{\sigma(\beta) \in \Sigma^*, \quad \beta \in T^* \text{ y } \exists M_f \in F \text{ tal que } \delta(M_0, \beta) \geq M_f\}$}
        \item{Tipo-T = $\{\sigma(\beta) \in \Sigma^*, \quad \beta \in T^* \text{ y } \delta(M_0, \beta) \text{ está definido } \forall t_j \in T, \quad \delta(\delta(M_0, \beta), t_j) \text{ no está definido } \}$}
        \item{Tipo-P = $\{\sigma(\beta) \in \Sigma^*, \quad \beta \in T^* \text{ y } \delta(M_0, \beta) \text{ está definido }\}$}
    \end{itemize}

    \hfill \break 
    $\beta$ es una secuencia de transiciones.

    $\delta: M \times \beta \mapsto M$
\end{frame}


%%%%%%%%%%%%%%%%%%%%%%%%%%%%%%%%%%%%%%%%%%%%%%%%%%%%%%

\begin{frame}{Clases de lenguajes RP}
    \begin{figure}[h]
        \includegraphics[scale=0.45]{graphics/rp_languages.png}
        \caption{12 clases de lenguajes reconocidos por Redes de Petri \citep{Peterson:81}}
    \end{figure}
\end{frame}

%%%%%%%%%%%%%%%%%%%%%%%%%%%%%%%%%%%%%%%%%%%%%%%%%%%%%%

\begin{frame}{Expresividad}
  \begin{columns}
    \column{.40\textwidth}
    \begin{figure}[h]
        \includegraphics[scale=0.45]{graphics/rp_af.png}
        \caption{Máquina de estado que computa el complemento a dos de un número binario \citep{Peterson:81}}
    \end{figure}
    \column{.60\textwidth}
        \begin{figure}[h]
            \includegraphics[scale=0.3]{graphics/rp_af2.png}
            \caption{RP equivalente a la máquina de estados\citep{Peterson:81}}
        \end{figure}
  \end{columns}
\end{frame}

%%%%%%%%%%%%%%%%%%%%%%%%%%%%%%%%%%%%%%%%%%%%%%%%%%%%%%
\begin{frame}{Expresividad}
    \begin{figure}[h]
        \includegraphics[scale=0.7]{graphics/rp_context_free.png}
        \caption{Red de petri reconocedora del lenguajes libre de contexto \citep{Peterson:81}}
    \end{figure}
\end{frame}

%%%%%%%%%%%%%%%%%%%%%%%%%%%%%%%%%%%%%%%%%%%%%%%%%%%%%%

\begin{frame}{Expresividad}
    \begin{figure}[h]
        \includegraphics[scale=0.7]{graphics/rp_context_sensitive.png}
        \caption{Red de petri reconocedora del lenguajes sensibles al contexto \citep{Peterson:81}}
    \end{figure}
\end{frame}

%%%%%%%%%%%%%%%%%%%%%%%%%%%%%%%%%%%%%%%%%%%%%%%%%%%%%%

\begin{frame}{Expresividad}
    \begin{figure}[h]
        \includegraphics[scale=0.6]{graphics/rp_chomsky.png}
        \caption{Lenguajes de RP en la clasificación de Chomsky \citep{Augusto:95}}
    \end{figure}
\end{frame}

%%%%%%%%%%%%%%%%%%%%%%%%%%%%%%%%%%%%%%%%%%%%%%%%%%%%%%%
%\input{examples/beamer-examples.tex}
%%%%%%%%%%%%%%%%%%%%%%%%%%%%%%%%%%%%%%%%%%%%%%%%%%%%%%% 
\begin{frame}[plain,allowframebreaks]
\frametitle{Referencias}

\insertBib

Esta presentación fue realizada con el siguiente template: \textit{\url{https://www.overleaf.com/13923686hnxdtdpgxfrf\#/53973373/}}

Para descargar esta presentación, dirigirse al siguiente sitio: \textit{\url{http://cor.to/pn-slides}}

\end{frame}
%%%%%%%%%%%%%%%%%%%%%%%%%%%%%%%%%%%%%%%%%%%%%%%%%%%%%%%

\end{document}
