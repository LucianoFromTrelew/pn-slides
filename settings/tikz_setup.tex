%%%%%%%%%%%%%%%%%%%%%%
%   TIKZ SETTINGS    % 
%%%%%%%%%%%%%%%%%%%%%%

\usepackage{tikz}
%%%%%%%%%%%%%%%%%%%%%%%%%%%%%%%%%%%%%
% tikz-qtree bugfix by Andrew Stacey  
\makeatletter
 
\def\unwind@subpic#1{%
% is #1 the current picture?
\edef\subpicid{#1}%
\ifx\subpicid\pgfpictureid
% yes, we're done
\else
% does #1 have a parent picture?
\expandafter\ifx\csname pgf@sh@pi@#1\endcsname\relax
% no, the original node was not inside the current picture
\pgf@xa=\pgf@x
\pgf@ya=\pgf@y
\pgfsys@getposition{\pgfpictureid}\pgf@shape@current@pos
\pgf@process{\pgfpointorigin\pgf@shape@current@pos}%
\advance\pgf@xa by-\pgf@x%
\advance\pgf@ya by-\pgf@y%
\pgf@process{\pgfpointorigin\subpic@parent@pos}%
\advance\pgf@xa by \pgf@x%
\advance\pgf@ya by \pgf@y%
\pgf@x=\pgf@xa
\pgf@y=\pgf@ya
\else
% yes, apply transform, save picture location, and move up to parent picture
\pgfsys@getposition{\csname pgf@sh@pi@#1\endcsname}\subpic@parent@pos%
{%
  \pgfsettransform{\csname pgf@sh@nt@#1\endcsname}%
  \pgf@pos@transform{\pgf@x}{\pgf@y}%
  \global\pgf@x=\pgf@x
  \global\pgf@y=\pgf@y
}%
\unwind@subpic{\csname pgf@sh@pi@#1\endcsname}%
\fi
\fi
}


\def\pgf@shape@interpictureshift#1{%
\def\subpic@parent@pos{\pgfpointorigin}%
\unwind@subpic{\csname pgf@sh@pi@#1\endcsname}%
}

\makeatother
% tikz-qtree bugfix by Andrew Stacey 
%%%%%%%%%%%%%%%%%%%%%%%%%%%%%%%%%%%%%

\tikzset{every tree node/.style={align=center,anchor=north}}	% to allow linebreaks
\usetikzlibrary{calc} % for positioning arrows with ($(t.center)-(1,0)$)
\usetikzlibrary{shapes,decorations}
\usetikzlibrary{backgrounds,fit}
\usetikzlibrary{arrows}
\usetikzlibrary{matrix}
\usetikzlibrary{positioning}
\usetikzlibrary{automata}
\usetikzlibrary{tikzmark}

% Define box and box title style (see http://www.texample.net/tikz/examples/boxes-with-text-and-math/)
\tikzstyle{mybox} = [draw=gray, very thick,
    rectangle, rounded corners, inner sep=10pt, inner ysep=17pt,yshift=3pt]
\tikzstyle{fancytitle} =[draw=gray, very thick, fill=white,
    rectangle, rounded corners, inner sep=5pt, inner ysep=5pt]
\tikzstyle{mydouble} = [double distance=1pt]
    
\tikzset{
    %Define standard arrow tip
    >=stealth',
    %Define style for boxes
    box/.style={
           rectangle,
           rounded corners,
           draw=black, very thick,
           text width=10em,
           minimum height=2em,
           text centered},
    % Define arrow style
    arrow/.style={
           ->,
           thick,
           	shorten <=2pt,
           shorten >=2pt,}
}

\newcommand\centertikz[1]{\tikz[baseline=(current bounding box.center)]{#1}}
\newcommand\tikzcenter{baseline=(current bounding box.center)}
\newcommand\tikztop{baseline=(current bounding box.north)}

\newcommand\tikztreeset[1]{\matrix [matrix of nodes,left delimiter=\{,right delimiter=\}](set){#1};}